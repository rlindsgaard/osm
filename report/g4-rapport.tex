\documentclass[titlepage]{article}
\usepackage[utf8]{inputenc}
\usepackage[danish]{babel}
\usepackage{amsmath}
\usepackage{ulem}
\usepackage{palatino}
\linespread{1.05}
\title{Operativsystemer og Multiprogrammering\\G-opgave 4}
\author{Ronni Elken Lindsgaard - 0911831791 \and
Hans-Kristian Bjerregaard - 0612862087 \and
Alexander Winther Uldall - 2908872013}
\date{16. marts, 2010}
\begin{document}
\maketitle
\newpage

\section{}

\section{}
  Her følger den generiske algoritme for at konvertere en logisk adresse til en fysisk adresse på en Intel Pentium processor.
  De første 4 punkter dækker segmenteringen mens de sidste 3 dækker over sideopslag.
  Navne i algoritmen refererer figurerne på side 346 (figurerne er unummererede) i SGG - 8. udgave.
  \begin{itemize}
    % segmentering
    \item Brug $g$ i selector til at bestemme om GDT eller LDT skal benyttes.
    \item Benyt $s$ til at slå base og limit op i GDT/LDT.
    \item Tjek at offset er mindre end limit, en fejl opstår hvis dette ikke er tilfældet.
    \item Beregn 32-bit lineær adresse ved at summere offset og base.
    % sideopslag for 4KB
    \item Benyt de første 10 bit ($p_1$) i den lineære adresse til at så op i sidekataloget for at finde den anvendte sidetabel.
    \item Benyt de efterfølgende 10 bit ($p_p$) til at slå op i den fundne sidetabel.
    \item Den fysiske adresse findes så ved at tage summen af resultatet i sidetabelen og den lineære adresses offset ($d$).
  \end{itemize}
  
  \subsection{}
    selector: 0x270, offset 0x10
    \\\\
    Selectoren i binær er 1001111000 hvilket giver $p = 00$, $g = 0$ og $s = 1001111$.
    Da $g$ er nul benyttes GDT.
    Decimalværdien af $s$ er 78 og på position 78 i GDT finder vi basen 0x803000 og limit 0x100.
    Da limit er større end offsetet i den logiske adresse kan vi fortsætte.
    Så beregnes den lineære adresse som summen af det logiske offset og basen hvilket giver 0x803010.
    \\\\
    Den binære værdi af den lineære adresse er 100000000011000000010000 hvilket giver $d = 000000010000$, $p_2 = 0000000011$ og $p_1 = 10$.
    Decimalværdien af $p_1$ er 2 hvilket benyttes til at slå op i sidekataloget.
    Decimalværdien af $p_2$ er 3 hvilet bruges til at slå op i den sidetabel fundet i sidekataloget.
    Den fysiske adresse er så sidens adresse plus $d$ hvilket giver: 0x07 + 000000010000 = 10111.
    
  \subsection{}
    selector: 0x278, offset 0xFA
    \\\\
    Selectoren i binær er 1001111000 hvilket giver $p = 00$, $g = 0$ og $s = 1001111$.
    Da $g$ er nul benyttes GDT.
    Decimalværdien af $s$ er 79 og på position 79 i GDT finder vi basen 0xC01000 og limit 0x80.
    Da limit er mindre end offsetet i den logiske adresse kan vi ikke fortsætte!
    
  \subsection{}
    selector: 0x10C, offset 0xFFF
    \\\\
    Selectoren i binær er 100001100 hvilket giver $p = 00$, $g = 1$ og $s = 100001$.
    Da $g$ er et benyttes LDT.
    Decimalværdien af $s$ er 33 og på position 33 i LDT finder vi basen 0xC00000 og limit 0x1000.
    Da limit er større end offsetet i den logiske adresse kan vi fortsætte.
    Så beregnes den lineære adresse som summen af det logiske offset og basen hvilket giver 0xC00FFF.
    \\\\
    Den binære værdi af den lineære adresse er 110000000000111111111111 hvilket giver $d = 111111111111$, $p_2 = 0000000000$ og $p_1 = 11$.
    Decimalværdien af $p_1$ er 3 hvilket benyttes til at slå op i sidekataloget.
    Decimalværdien af $p_2$ er 0 hvilet bruges til at slå op i den sidetabel fundet i sidekataloget.
    Den fysiske adresse er så sidens adresse plus $d$ hvilket giver: 0x43 + 000000010000 = 1010011.
\end{document}
