\documentclass[titlepage]{article}
\usepackage[utf8]{inputenc}
\usepackage[danish]{babel}
\usepackage{amsmath}
\usepackage{ulem}
\usepackage{palatino}
\linespread{1.05}
\title{Operativsystemer og Multiprogrammering\\G-opgave 4}
\author{Ronni Elken Lindsgaard - 0911831791 \and
Hans-Kristian Bjerregaard - 0612862087 \and
Alexander Winther Uldall - 2908872013}
\date{16. marts, 2010}
\begin{document}
\maketitle
\newpage
\section{Implementering af {\tt ptable}}
Vores struktur består af en pointer til et array. Der skal være plads til $2^{10} = 1024$
elementer. Denne struktur bruges både til den ydre og den indre tabel. Teknisk set allokerer vi et
2-dimensionelt array, men skulle vi allokere alle pladser med det samme ville dette sluge rigtig
meget plads som vi sandsynligvis ikke har brug for. Vi allokerer i stedet de indre sider {\it on
demand} således at en side først oprettes når den bliver tilgået. 

\subsection{ptable\_set}
Først hentes den indre og ydre page med bit-operatorer, samt ofsettet. Siden bliver sat til {\tt
NULL} hvis page argumentet er {\tt NULL} og ellers oprettes siden (hvis den ikke allerede
eksisterer) og bliver tildelt en værdi. Offset bruges ikke da den jo først skal bruges på den
specifkke page.
\subsection{ptable\_get}
Den eneste logik der er brug for i {\tt ptable\_get} er et tjek om den ydre side eksisterer (ellers
risikerer vi segfault ved næste operation) og hvis den gør, returneres den page som den indre tabel
peger på. Det kan lade sig gøre at returnere direkte da den vil være sat til {\tt NULL} hvis den
ikke har fået tildelt en værdi grundet {\tt calloc} kaldet i {\tt ptable\_set}.
\subsection{Korrekthed og kørsel af tests}
I opgavebeskrivelsen står at vi skal se bort fra 64 bit adresser mv. vi har derfor valgt at "antage"
at programmet køres på en 32-bits maskine.
Vi tester opførslen for 2 forskellige sider således at vi sikrer os at ingen af de operationerne
påvirker dem begge (f.eks. free på den ene også free'er den anden)
Til at starte med testes også for om dataene virkelig er unitialiserede.

En sidste ting der kan testes for er om dataene også behandles forskelligt når kun den indre tabel
er forskellig. Det er tydeligt at se i koden og vi har derfor valgt ikke at gøre dette.

Vi har en memory leak som ikke er ordnet. Dette skyldes højst sandsynligt at vi allokerer plads i
{\tt ptable\_set} men at vi aldrig får deallokeret dette. En måde at gøre det på, er ved at holde
styr på hukommelsen i tabel-strukturen og free'e hele den indre tabel når det ses at denne er tom.
\section{Sideerstatning}

\begin{tabular}{l | c | c | c | c}

t+1\\
Sideplads    & 0 & 1 & 2 & 3\\
Side         & 5 & 8 & 16 & 4\\
Reference bit & 1 & 0 & 1 & 1\\\\  
\end{tabular}

her sker ingen ændring da vi kun slår en side op.\\

\begin{tabular}{l | c | c | c | c}
t+2\\
Sideplads    & 0 & 1 & 2 & 3\\
Side         & 5 & 6 & 16 & 4\\
Reference bit & 0 & 1 & 0 & 0\\\\
\end{tabular}

side 6 eksistere ikke og swappes derfor ind.\\
det nuværende offer er sideplads 2, efter kørsel er denne sideplads 2.\\

\begin{tabular}{l | c | c | c | c}
t+3\\
Sideplads    & 0 & 1 & 2 & 3\\
Side         & 5 & 6 & 16 & 4\\
Reference bit & 0 & 1 & 1 & 0\\\\
\end{tabular}

vi sætter control bit på side 16 til 1 da vi læser den igen.\\


\begin{tabular}{l | c | c | c | c}
t+4\\
Sideplads    & 0 & 1 & 2 & 3\\
Side         & 5 & 6 & 16 & 8\\
Reference bit & 0 & 1 & 0 & 1\\\\
\end{tabular}

første page fault.\\
side 8 eksistere ikke og swappes derfor ind.\\
det nuværende offer er sideplads 2, efter kørsel er denne sideplads 0.\\

\begin{tabular}{l | c | c | c | c}
t+5\\
Sideplads    & 0 & 1 & 2 & 3\\
Side         & 5 & 6 & 16 & 8\\
Reference bit & 1 & 1 & 0 & 1\\\\
\end{tabular}

vi sætter control bit på side 5 til 1 da vi læser den igen.\\

\begin{tabular}{l | c | c | c | c}
t+6\\
Sideplads    & 0 & 1 & 2 & 3\\
Side         & 5 & 6 & 9 & 8\\
Reference bit & 0 & 0 & 1 & 1\\\\
\end{tabular}

side 9 eksistere ikke og swappes derfor ind.\\
det nuværende offer er sideplads 0, efter kørsel er denne sideplads 3.\\

\begin{tabular}{l | c | c | c | c}
t+7\\
Sideplads    & 0 & 1 & 2 & 3\\
Side         & 10 & 6 & 9 & 8\\
Reference bit & 1 & 0 & 1 & 0\\\\
\end{tabular}

side 10 eksistere ikke og swappes derfor ind.\\
det nuværende offer er sideplads 3, efter kørsel er denne sideplads 1.\\

\begin{tabular}{l | c | c | c | c}
t+8\\
Sideplads    & 0 & 1 & 2 & 3\\
Side         & 10 & 6 & 9 & 8\\
Reference bit & 1 & 1 & 1 & 0\\\\
\end{tabular}

vi sætter control bit på side 6 til 1 da vi læser den igen.\\


\begin{tabular}{l | c | c | c | c}
t+9\\
Sideplads    & 0 & 1 & 2 & 3\\
Side         & 10 & 6 & 9 & 7\\
Reference bit & 1 & 0 & 0 & 1\\\\
\end{tabular}

side 7 eksistere ikke og swappes derfor ind.\\
det nuværende offer er sideplads 1, efter kørsel er denne sideplads 0.\\

\begin{tabular}{l | c | c | c | c}
t+10\\
Sideplads    & 0 & 1 & 2 & 3\\
Side         & 10 & 8 & 9 & 7\\
Reference bit & 0 & 1 & 0 & 1\\\\
\end{tabular}

anden page fault da side 8 ikke eksistere på den gamle plads.\\
side 8 eksistere ikke og swappes derfor ind.\\
det nuværende offer er sideplads 0, efter kørsel er denne sideplads 2.\\

\begin{tabular}{l | c | c | c | c}
t+11\\
Sideplads    & 0 & 1 & 2 & 3\\
Side         & 10 & 8 & 3 & 7\\
Reference bit & 0 & 1 & 1 & 1\\\\
\end{tabular}

side 3 eksistere ikke og swappes derfor ind.\\
det nuværende offer er sideplads 2, efter kørsel er denne sideplads 3.\\


\begin{tabular}{l | c | c | c | c}
t+12\\
Sideplads    & 0 & 1 & 2 & 3\\
Side         & 10 & 8 & 3 & 7\\
Reference bit & 0 & 1 & 1 & 1\\\\
\end{tabular}

intet sker der læses blot\\




\end{document}
