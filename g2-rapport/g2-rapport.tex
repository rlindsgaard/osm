\documentclass[titlepage]{article}
\usepackage[utf8]{inputenc}
\usepackage[danish]{babel}
\usepackage{amsmath}
\usepackage{ulem}
\usepackage{palatino}
\linespread{1.05}
\title{Operativsystemer og Multirogrammering \\G-opgave 2}
\author{Ronni Elken Lindsgaard - 0911831791 \and
Hans-Kristian Bjerregaard - 0612862087 \and
Alexander Winther Uldall - 2908872013}
\date{23. februar, 2010}
\begin{document}
\maketitle
\newpage
\section{En ring af tråde}
I ring.h er defineret to datastrukturer; data og baton. Baton er den stafet som bliver delt mellem trådene, helt specifikt er den et specifikt adresserum som de alle kan tilgå efter tur.
Data definerer et datasegment for den specifikke tråd som fortæller dens id, dens condition, hvem der er den næste i rækken og hvor den skal finde sit delte adresserum.
Ved initialisering bliver datasegmenter og stafetten skabt og der sørges for at alle tråde er låst til at starte med.

Inde i tråden låser tråden først mutexen hvorefter den finder ud af om det egentlig er dens tur til at køre ved at teste om dens id er lig det id der skal køre hvis ikke sender den signal til den næste tråd der så forsøger på det samme. Dette fortsætter den med hver gang det ikke er dens egen tur.

Hvis det rent faktisk er den tråd der skal køre starter den med at sætte current til den næste tråd i ringen og sende et signal, dette gør at den stiller sig i "venteposition" og begynder så snart mutexen unlocker.

Der tælles nu om der er nået en omgang og hvor mange omgange der er kørt, hvis der er kørt det antal omgange specificeret i NUM\_RUNS bliver stafettens stop-signal sat og værdien bliver kopieret over i en lokal variabel (for at undgå race conditions)

Til sidst udføres noget specifikt arbejde (såsom at printe en linie ud) og mutexen bliver aflåst.

Efter alle tråde er kørt færdig bliver de joinet hvorefter allokeret hukommelse bliver rydtet op.
\end{document}

